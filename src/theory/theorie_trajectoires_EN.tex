\documentclass [12pt]{article}
%\usepackage{ae,lmodern} % ou seulement l'un, ou l'autre, ou times etc.
\usepackage{xcolor} 
\usepackage{wrapfig}
\usepackage[francais]{babel}
\usepackage[utf8]{inputenc}
%\usepackage[T1]{fontenc}
\usepackage{amsmath}
\usepackage{graphicx}
\usepackage[export]{adjustbox}
\usepackage{pdfpages}
\usepackage[pdfnewwindow=true,
    pdftex,                %
    bookmarks         = true,%     % Signets
    bookmarksnumbered = true,%     % Signets num�©rot�©s
    pdfpagemode       = None,%     % Signets/vignettes ferm�© �  l'ouverture
    pdfstartview      = FitH,%     % La page prend toute la largeur
    pdfpagelayout     = SinglePage,% Vue par page
    colorlinks        = true,%     % Liens en couleur
    urlcolor          = blue,%  % Couleur des liens externes
		linkcolor         = blue,%   % Couleur des liens internes
    pdfborder         = {0 0 0}%   % Style de bordure : ici, pas de bordure
    ]{hyperref}%
\parindent=0mm
\textwidth=18cm
\textheight=25cm
\voffset=-3.5cm
\hoffset=-2cm
\def\ph#1{\hskip #1}
\def\pv#1{\vskip #1}

\font\petit=cmbx10 scaled 700
\font\bidon=cmbx10 scaled 1200
\font\titre=cmbx10 scaled 1100
\font\defi=cmbxti10 scaled 1200
\font\defibis=cmbxti10 scaled 1000
\font\defilarge=cmbxti10 scaled 1500
\font\large=cmbx10 scaled 1300

\definecolor{bleu}{rgb}{0, 0, 1}


\begin{document}

\ph 150mm {\petit J.P.Cordoni} \pv -2mm

\ph 150mm {\petit 2023/01/25}


	\begin{wrapfigure}{r}{0.5\textwidth}
        %centering
				%\begin{center}
	   \includegraphics[scale=0.4]{../Images/scientifique_deconcerte_EN.jpg}	
		\vspace{-5cm}
		\centering
		%\end{center}
		%\caption{ }
    %\label{ }
	  \end{wrapfigure}
 
\centerline{\defilarge Trajectories Theory and Cosmogravity } \pv 10mm

              
   \hyperlink{ancre01}{\titre I - Schwarzschild Metrics}

\ph 5mm \hyperlink{ancre02}{\petit I.1 External Metric}

\ph 5mm \hyperlink{ancre03}{\petit I.2 Internal Metric}

\hyperlink{ancre04}{\titre II - Kerr Metric }

\hyperlink{ancre05}{\titre III - Appendix}

\ph 5mm \hyperlink{ancre06}{\petit III.1 Euler-Lagrange equations}

\ph 5mm \hyperlink{ancre07}{\petit III.2  Metrics and geodesics }

\ph 5mm \hyperlink{ancre08}{\petit III.3  Application }

\ph 5mm \hyperlink{ancre09}{\petit III.4  Runge-Kutta method of order  4}

\ph 5mm \hyperlink{ancre10}{\petit III.5  Physical speed  }

\ph 5mm \hyperlink{ancre11}{\petit III.6  Physical speed in Schwarzschild metric }

\ph 5mm \hyperlink{ancre12}{\petit III.7  Physical speed in Kerr metric }

\pv 5mm



\hypertarget{ancre01}{\centerline{\large\textcolor{red}{ I - Schwarzschild Metrics}} }\pv 5mm
            \hypertarget{ancre02}{{\bidon I.1 External Metric}} \pv3mm       



              We wish to study the motion of a particle (massive or photon) in the external gravitational field of a centro-symmetric mass, without rotation,
			placed at the origin of spatial coordinates $r, \theta,\varphi$ and time $t$.
			The model is a static space-time with spherical symmetry of metric :
			$$ds^2= -e^{\lambda(r)}dr^2 -r^2(d\theta^2+sin^2\theta d\varphi^2)+e^{\nu(r)} c^2dt^2$$
			In  {\bf general relativity}, for a mass $M$  without rotation, spherically symmetric, placed at the origin of the coordinates, in an empty space-time, the solution of Einstein's equations is:
		\pv 3mm
			\centerline{$\lambda+\nu=0 \ \ \ \ \  \displaystyle e^\nu=1-\frac{r_s}{r}$ \ph 5mm   with \ph 5mm  \fcolorbox{black}{green}{$r_s=2\,\displaystyle\frac{GM}{c^2}$}}
			\pv 3mm									
			hence the expression of   Schwarzschild's  metric :
	\pv 3mm
			\centerline {\fcolorbox{black}{green}{${ ds^2= -\displaystyle\frac{dr^2}{ 1-\displaystyle\frac{r_s}{r}} - r^2(d\theta^2 + sin^2 \theta d\varphi^2) +\left( 1-\frac{r_s}{r} \right) c^2dt^2 }$}}
\pv 3mm   
  $c$  the speed of light  and  G  the gravitational constant .
								
                 $r_s$  is called \og Schwarzschild rayon\fg \, or \og black hole horizon\fg. It represents the limit of the region from which light and matter cannot escape.
         \pv 3mm    
			  This Schwarzschild solution is of remarkable importance since it is also the only solution to Einstein's equations outside of any spherically symmetric body of mass M, without rotation.
			This metric is therefore not limited to describe only black holes, it is also valid outside a star, planet, or any other spherically symmetric body without rotation.  
			This result is known as the Birkhoff theorem (Gauss theorem in Newtonian gravitation).
			 

             
                The trajectories of a free particle are the geodesics of metric. 
								
								By reason of symmetry   the trajectory is flat  (we will take $\theta=\displaystyle\frac{\pi}{2}$).
             
			\pv 20mm
          {\defi I.1.1 Massive particle} 
       \pv 2mm    
		  		    The first integrals found using the Lagrangian $\mathcal L=\displaystyle{ds\over d\tau}$ are written:
       
         $$(S^e_{pm}1)\ph 3mm \displaystyle{dt\over d\tau}(r)={ E_e\over \displaystyle {1- {r_s \over r}}}\ph 5mm  and \ph 5mm \ph 3mm (S^e_{pm}2)\ph 3mm {d\varphi\over d\tau}(r) = {c\,L_e\over r^2}$$
       
       
        With two integration constants: $E_e$ dimensionless and $L_e$ one length. These functions reported in the expression $ds=c\,d\tau$ involve:

      		$$(S^e_{pm}3)\ph 3mm \displaystyle(\frac{dr}{d\tau})^2+ V^e_{Spm}=c^2E_e^2 \ph 3mm  where \ph 3mm V^e_{Spm}(r)= c^2(1 -\frac{r_s}{r}) (1+\frac{L_e^2}{r^2}) $$

        These relations verify the equations of the geodesics. 

		 By deriving with respect to $\tau$ the equation $(S^e_{pm}3)$ and for non-circular orbits, we deduce the following differential equation used to make the simulation :
			$$(S^e_{pm}4)\ph 3mm  \frac{d^2r}{d\tau^2}= -\frac{1}{2} \frac{dV^e_{Spm}}{dr}=\frac{c^2}{2r^4}(-r_sr^2+2rL_e^2-3r_sL_e^2)=f^e_{Spm}(r) $$ 


		The study of the function $V^e_{Spm}$ allows us to deduce that there can exist two circular orbits of radii :
		$$\displaystyle\frac{L_e}{r_s}\left(L_e+\sqrt{L_e^2-3r_s^2}\right)~~(stable)~~~ and~~~ \displaystyle\frac{L_e}{r_s}\left(L_e-\sqrt{L_e^2-3r_s^2}\right)~~ (unstable).$$

		The equation $S^e_{pm}4$ is solved numerically by the Runge-Kutta method.
		
		\hyperlink{A2massif} {Initial values III.6-A.2}
		       
			 The observer who has stayed away from the black hole sees his colleague moving more and more slowly and eventually freezing when he reaches the $r_s$ horizon.
			 A traveler who falls into the black hole arrives at the center (r=0) in a finite time while his colleague has the impression that he remains frozen on the horizon.
			 (and, in practice, disappears because of the spectral shift). 
             
\pv 1.5mm

{\defi I.1.2 Photon}
          
    \pv 1.5mm          
         We keep the relations in $t$ and  $\varphi$  with a different parameter $\lambda$ of the proper time $\tau$ since for a
photon we have always $d\tau$ = 0.
\pv -2mm
		$$(S^e_{ph}1)\ph 3mm \displaystyle{dt\over d\lambda}(r)={ E_e\over \displaystyle {1- {r_s \over r}}}\ph 5mm  and \ph 5mm \ph 3mm (S^e_{ph}2)\ph 3mm {d\varphi\over d\lambda}(r) = {c\,L_e\over r^2}$$
		These functions reported in the expression $ds=0$ involve:
\pv -2mm
      		$$(S^e_{ph}3)\ph 3mm \displaystyle(\frac{dr}{d\lambda})^2+ V^e_{Sph}=c^2E_e^2 \ph 3mm  where \ph 3mm V^e_{Sph}(r)= c^2(1 -\frac{r_s}{r})\frac{L_e^2}{r^2} $$

			These relations verify the equations of the geodesics.

		By deriving with respect to $\lambda$ the equation $(S^e_{ph}3)$ and for non-circular orbits, we deduce the following differential equation used to make the simulation :
			$$(S^e_{ph}4)\ph 3mm  \frac{d^2r}{d\lambda^2}= -\frac{1}{2} \frac{dV^e_{Sph}}{dr}=\frac{c^2}{2r^4}(2rL_e^2-3r_sL_e^2)=f^e_{Sph}(r) $$ 


		The study of the function $V^e_{Sph}$ shows that there is an unstable circular orbit of radius $\displaystyle\frac{3}{2}r_s$.  
		
		The equation $S^e_{ph}4$ is solved numerically by the Runge-Kutta method.  
		
		\hyperlink{A1photon} {Initial values III.6-A.1}
      
			\pv 5mm
				
			\hypertarget{ancre03}{{\bidon I.2 Internal Metric}}
     \pv 3mm 
           	
		  
		 We wish to study the motion of a particle (subject only to gravitation) inside a {\bf constant density}, centro-symmetric, non-rotating object. 
			 The mass is placed at the origin of spatial coordinates $r, \theta,\varphi$ and time $t$. 
		<The model is a static space-time with spherical symmetry. 
			 The solution of Einstein's equations gives the metric :  
	\pv 3mm
			  \centerline{\fcolorbox{black}{green}{$ {ds^2= -\alpha(r) dr^2\ -\ r^2(d{\theta}^2+\sin^2\theta\, d{\varphi}^2)\ +\ \beta(r)^2\,c^2dt^2 }$}}
			\pv 3mm
			 \centerline {  \ph 18mm where \ph 5mm \fcolorbox{black}{green}{$\alpha (r)= 1-\displaystyle\frac{r^2 r_s}{R^3}$} \ph 5mm\fcolorbox{black}{green}{$\beta (r)=\frac{3}{2}\ \sqrt{\displaystyle 1-\frac{r_s}{ R}}\ -
             \ \frac{1}{2}\ \sqrt{\displaystyle 1-\frac{r^2 r_s}{R^3}}$} \ph 5mm  $ R$ object radius }
			  \pv 3mm					  
                The trajectories of a free particle are the geodesics of metric. 
								
								By reason of symmetry   the trajectory is flat  (we will take $\theta=\displaystyle\frac{\pi}{2}$).
              \pv 1.5mm
			 For the establishment of this metric see \href{./documents/Andrillat.pdf#page=1 }{\textcolor{bleu}{Henri Andrillat - Introduction \`a l'\'etude des cosmologies} }. 
			
			 
	  	\pv 2mm      
	  {\defi I.2.1 Massive particle} 
		
		\begin{figure}[!h]
    %\centering
	   \includegraphics[width=150mm,height=60mm,center]{../Images/masse_non_baryonique_en.jpg}	
		%\caption{ }
    %\label{ }
	  \end{figure}
		
		          
	  	  	 	 	 
		    The first integrals found using the Lagrangian $\mathcal L=\displaystyle{ds\over d\tau}$ are written:
       
      
        $$(S^i_{pm}1) \ph 3mm  \beta(r)^2\displaystyle{dt\over d\tau}(r)=E_i \ph 5mm  and \ph 5mm \ph 3mm (S^i_{pm}2)\ph 3mm {d\varphi\over d\tau}(r) = {c\,L_i\over r^2}$$    
      
       
        With two integration constants: $E_i$ dimensionless and $L_i$ one length. These functions reported in the expression $ds=c\,d\tau$ involve:

      	$$ (S^i_{pm}3)\ph 5mm ({dr\over d\tau})^2\,+\,V^i_{Spm}(r) = c^2 E_i^2 \ph 3mm  where  \ph 3mm  V^i_{Spm}(r)=c^2 E_i^2 - \ c^2\alpha (r)\,\left [\frac{E_i^2}{\beta (r)^2}-\frac{L_i^2}{r^2}-1\right ]$$
	
					  				
		 	 These relations verify the equations of the geodesics. 

		 By deriving with respect to $\tau$ the equation $(S^i_{pm}3)$ and for non-circular orbits, we deduce the following differential equation used to make the simulation :
		\pv-2mm
		
			$$(S^i_{pm}4)\ph 3mm  \frac{d^2r}{d\tau^2}=-\frac{1}{2}\,\frac{dV^i_{Spm}}{dr} = -\frac{c^2\,r\,r_s}{R^3}\left [\frac{E_i^2}{\beta (r)^2}-\frac{L_i^2}{r^2}-1\right ]+
            \frac{c^2\,\alpha(r)}{2}\left [\frac{-E_i^2\,r\,r_s}{\beta(r)^3\sqrt{\alpha(r)}R^3}+2\frac{L_i^2}{r^3}\right ]=f^i_{Spm}(r) $$ 
\pv -2mm
		
		The equation $S^i_{pm}4$ is solved numerically by the Runge-Kutta method.  
	   
		 \hyperlink{B2massif}{Initial values III.6-B.2}
  	  \pv 50mm 
	  
	  	  
	  {\defi I.2.2 Photon}
       \pv 2mm   
		   
     		   We keep relations in $t$ and $\varphi$ :  

		$$(S^i_{ph}1)\ph 3mm \beta(r)^2\displaystyle{dt\over d\lambda}(r)=E_i\ph 5mm  and \ph 5mm \ph 3mm (S^i_{ph}2)\ph 3mm {d\varphi\over d\lambda}(r) = {c\,L_i\over r^2}$$
		 These functions reported in the expression $ds=0$ involve:	  
	 $$ ({dr\over d\lambda})^2\,+\,V^i_{Sph}(r) = c^2 E_i^2 \ph 3mm  and \ph 3mm  V^i_{Sph}(r)=c^2 E_i^2 - \ c^2\alpha (r)\,\left [\frac{E_i^2}{\beta (r)^2}-\frac{L_i^2}{r^2}\right ]$$
	 
	  By deriving with respect to $\lambda$ the equation $(S^i_{ph}3)$ and for non-circular orbits, we deduce the following differential equation used to make the simulation :
			$$(S^i_{ph}4)\ph 3mm  \frac{d^2r}{d\lambda^2}=-\frac{1}{2}\,\frac{dV^i_{Sph}}{dr} = -\frac{c^2\,r\,r_s}{R^3}\left [\frac{E_i^2}{\beta (r)^2}-\frac{L_i^2}{r^2}\right ]+
			\frac{c^2\,\alpha(r)}{2}\left [\frac{-E_i^2\,r\,r_s}{\beta(r)^3\sqrt{\alpha(r)}R^3}+2\frac{L_i^2}{r^3}\right ]=f^i_{Sph}(r) $$ 
	
		 
	The equation $S^i_{ph}4$ is solved numerically by the Runge-Kutta method.  
	  
	   
	 \hyperlink{B1photon} {Initial values  III.6-B.1} 
	  	\pv 7mm
				
		    \hypertarget{ancre04}{ \centerline {\large\textcolor{red}{ II - Kerr Metric } }} \pv 2mm
      {\bidon II.1 General Theory}  
	 
       \pv 5mm    
       We wish to study the motion of a particle (massive or photon) in the gravitational field of a centro-symmetric, rotating mass,
        placed at the origin of the coordinates . 
	
		In  {\bf general relativity} , for a  mass $M$ in rotation, spherically symmetric, placed at the origin of the Boyer-Lindquist coordinates $r, \theta,\varphi ,t$, in an empty space-time, 		the solution of Einstein's equations is the  Kerr metric :
       
			\pv 5mm
	\centerline{\fcolorbox{black}{green}{ $\displaystyle ds^2={-\rho^2\over\Delta}dr^2-\rho^2 d\theta^2-(r^2+a^2+{r_s\,r\,a^2\over\rho^2}\,\sin^2\theta)\sin^2\theta\,d{\varphi}^2
        +{2\,r_s\,r\,a\over\rho^2}\,\sin^2\theta\,c\,dt\,d\varphi+(1-\displaystyle {r_s\,r\over \rho^2})\,c^2dt^2 $} }
      \pv 5mm 
     
			\centerline {where \ph 5mm \fcolorbox{black}{green}{$\rho^2(r)=r^2+a^2\cos^2\theta$} \ph 5mm \fcolorbox{black}{green}{$\Delta(r)=r^2-r_s\,r+a^2$} \ph 5mm
			\fcolorbox{black}{green}{$a=\displaystyle{J\over c\,M}$} \ph 5mm (J\ angular\  momentum)}
              

         \pv 3mm
          Unlike Schwarzschild metric, there is no equivalent of Birkhoff's theorem in Kerr metric. This geometry therefore only describes \fcolorbox{black}{green}{rotating black holes}, not space-time
		  external to other objects such as rotating stars or planets.

         The trajectories of a free particle are the geodesics of the metric.   
				
				{\bf Only flat trajectories will be studied }  (with $\theta=\displaystyle\frac{\pi}{2}$).

			   The event horizon  corresponds to the change of sign of $g_{rr}$ , i.e. to the solutions of the equation $\Delta=0$. 	 
			If $\displaystyle a<{r_s\over 2}$ we get two values :
	\pv 2mm		
			\centerline{\fcolorbox{black}{green}{$R_{H+}=\displaystyle\frac{r_s+\sqrt{r_s^2-4a^2}}{2}$} \ph 3mm and \ph 3mm \fcolorbox{black}{green}{$R_{H-}=\displaystyle\frac{r_s-\sqrt{r_s^2-4a^2}}{2}$} 
				\ph 3mm		so \ph 3mm \fcolorbox{black}{green}{$\Delta(r)=(r-R_{H+})(r-R_{H-})$} } 
			 \pv 5mm				
			  The domain between $r_s$ and $R_{H+}$ is called  ergoregion  (in Schwarzschild metric there is no ergoregion $a=0$ and   $R_{H+}=r_s$) see
			\href{documents/Extrait_Gourgoulhon_relatM2-page133.pdf }{\textcolor{bleu}{\'Eric Gourgoulhon - Relativit\'e g\'en\'erale} }  .
			
     \pv 50mm
      {\defi II.2 Massive particle}
			
				\begin{figure}[!h]
    %\centering
	   \includegraphics[width=150mm,height=60mm,center]{../Images/masse_Kerr_en.jpg}	
		%\caption{ }
    %\label{ }
	  \end{figure}  	
        \pv -2mm
               The first integrals found using the Lagrangian $\mathcal L=\displaystyle{ds\over d\tau}$ are written:
              \pv -4mm
        $$(K_{pm}1)\ph 3mm \displaystyle{dt\over d\tau}(r)={1\over\Delta(r)}\,\left[\,(r^2+a^2+{r_s\over r}\,a^2)\,E-{r_s\,a\over r}\,L\,\right]\ph 3mm  and\ph 3mm 
        (K_{pm}2)\ph 3mm \displaystyle{d\varphi\over d\tau}(r) = {c\over\Delta(r)}\,\left[\,{r_s\,a\over r}\,E+(1-{r_s\over r})\,L\,\right]$$
         With two integration constants: $E$ dimensionless and $L$ one length.
            
        These functions reported in the expression $ds=c\,d\tau$ involve:
       \pv -6mm
        $$(K_{pm}3)\ph 3mm \displaystyle({dr\over d\tau})^2+V_{Kpm}=c^2E^2 \ \ \ \ \ where \ \ \ \ \ V_{Kpm}(r)=c^2-{r_s\over r}\,c^2-{c^2\over r^2}\,(\,a^2\,(E^2-1)-L^2\,)-{r_s\,c^2\over r^3}\,(L-a\,E)^2 $$
               These relations verify the equations of the geodesics.

		  Deriving from $\tau$ the equation $(K_{pm}3)$ and for non-circular orbits, we deduce the following differential equation used to make the simulation : 
			\pv -4mm
		$$(K_{pm}4)\ph 3mm \frac{d^2r}{d\tau^2}=-\frac{1}{2}\frac{dV_{Kpm}}{dr}\  =\ -\frac{c^2}{2\,r^4}\,\left[r_s\,r^2+2\,r\,(a^2(E^2-1)-L^2)+3\,r_s\,(L-aE)^2\right]\ =f_{Kpm}(r)$$  
		\pv -4mm
	
		The equation $K_{pm}4$ is solved numerically by the Runge-Kutta method.
		 
		   \hyperlink{Kerrmassif} {Initial values III.7-A.2}
\pv 2mm
          {\defi II.3 Photon}
  \pv 2mm     
   
           We keep relations in $t$ and $\varphi$  :  \pv -5mm
              
        $$(K_{ph}1)\ph 3mm \displaystyle{dt\over d\lambda}={1\over\Delta(r)}\,\left[\,(r^2+a^2+{r_s\over r}\,a^2)\,E-{r_s\,a\over r}\,L\,\right]\ph 5mm and \ph 5mm 
         (K_{ph}2)\ph 3mm \displaystyle{d\varphi\over d\lambda} = {c\over\Delta(r)}\,\left[\,{r_s\,a\over r}\,E+(1-{r_s\over r})\,L\,\right]$$
       
        These functions reported in the expression $ds=0$ involve :
             
        $$(K_{ph}3)\ph 3mm  \displaystyle({dr\over d\lambda})^2+V_{Kph}(r)=c^2\,E^2 \ \ \ where\ \ \  V_{Kph}(r)=\displaystyle -{c^2\over r^2}\,(\,a^2\,E^2-L^2\,)-{r_s\,c^2\over r^3}\,(L-a\,E)^2 $$
       
       
        These relations verify the equations of the geodesics.

		By deriving with respect to $\lambda$ the equation $(K_{ph}3)$ and for non-circular orbits, we deduce the following differential equation used to make the simulation :
		
      $$(K_{ph}4)\ph 3mm \frac{d^2r}{d\lambda^2}=-\frac{1}{2}\frac{dV_{Kph}}{dr}=-\frac{c^2}{2\,r^4}\,\left[2\,r\,(a^2E^2-L^2)+3\,r_s\,(L-aE)^2\right]=f_{Kph}(r)$$
			
	\pv 50mm
		   	There are two unstable circular orbits of radii (see \href{documents/Bardeen_1972_ApJ.pdf}{\textcolor{bleu}{ James M. Bardeen} })   :
			
		$$r_s\{ 1+\cos [\frac{2}{3} \arccos(\frac{2a}{r_s})]\}\ph 3mm  and \ph 3mm  r_s\{ 1+\cos [\frac{2}{3} \arccos(\frac{-2a}{r_s})]\}$$

      The equation $K_{ph}4$ is solved numerically by the Runge-Kutta method.
			
			\hyperlink{Kerrphoton} {Initial values III.7-A.1}
			\pv 10mm
			
							\hypertarget{ancre05}{\centerline{\large\textcolor{red}{ III - Appendix } }} \pv4mm
            \hypertarget{ancre06}{{\bidon III.1 Euler-Lagrange equations }} \pv3mm  

					Let $\mathcal L\,(x^1(\lambda),x^2(\lambda),...,x^n(\lambda)\, , \,\dot x^1(\lambda),\dot x^2(\lambda),...,\dot x^n(\lambda))$
			a function of 2n independent variables 
			
			where $\dot x^i= \displaystyle{dx^i \over d\lambda}$. 
			
			So, the value of the integral $\displaystyle\int_{\lambda_a}^{\lambda_b} \mathcal L \,(\lambda)\,d\lambda$ is extreme for curves $\{x^i(\lambda)\}_{i\in\{1,2...,n\}}$ 
			\pv 2mm
			that verify Euler-Lagrange's equations :
		$$ {d\over d\lambda}\,{\partial\mathcal L \over \partial\dot x^k} -{\partial\mathcal L\over \partial x^k}\,=\,0  \ph 3mm  \forall k\in\{1,2,...n\}$$
		 
\pv 3mm
		\hypertarget{ancre07}{{\bidon III.2  Metrics and geodesics }} \pv3mm
		For an exhaustive study see \href{https://luth.obspm.fr/~luthier/gourgoulhon/bh16/bholes.pdf}{\textcolor{bleu}{Eric Gourgoulhon - Geometry and physics of black holes } } 
		
		and \href{https://luth.obspm.fr/~luthier/gourgoulhon/fr/master/relatM2.pdf}{\textcolor{bleu}{Eric Gourgoulhon - Relativit\'e g\'en\'erale } } .
		\pv 3mm
		{\defi Metrics} \pv2mm  
				  In a space-time, we represent by $ds^2$ the infinitesimal interval between two events marked by the coordinates $(x^1, x^2,x^3,x^4=ct)$ and $(x^1+dx^1, x^2+dx^2,x^3+dx^3,ct+cdt)$ where

		$$\ \ ds^2\,=\,\displaystyle \sum ^4_{i=1}\,\sum ^4_{j=1}\,g_{ij}(x^1,x^2,x^3,x^4)\,dx^i\,dx^j\,=\,g_{ij}\,dx^i\,dx^j \ \ (with\  Einstein\  summation\  notation) $$
		
		By abuse of language, we will speak of \og distance \fg \,between these two events.
		The 4x4 symmetric matrix of the 16 functions $g_{ij}$ (coefficients of the metric) allowing an inverse (whose coefficients are noted $g^{\alpha\beta}$) at any point where the metric is defined.
		\pv 2mm
		Basic properties of this \og distance \fg \, :

		  $\ \ \ $     - It's an invariant for any change in coordinates
			$$\ \ \ ds^2\,=\,\displaystyle \sum ^4_{i=1}\,\sum ^4_{j=1}\, g_{ij}(x^1,x^2,x^3,x^4)\,dx^i\,dx^j\,=
		\,\displaystyle \sum ^4_{i=1}\,\sum ^4_{j=1}\,\bar g_{pq}(y^1,y^2,y^3,y^4)\,dy^p\,dy^q \ \ \ \ \ \ \ \ \ (g_{ij}\,dx^i\,dx^j\,=\,\bar g_{pq}\,dy^p\,dy^q)  $$

		    $\ \ \ $   - It's spelled $ds=c\,d\tau$ where $\tau$ is the eigentime measured (by clock) between events $(x^1, x^2,x^3,x^4)$ and $(x^1+dx^1, x^2+dx^2,x^3+dx^3,x^4+dx^4)$ 
				\pv 2mm
		   $\ \ \ $   - For a photon, you always have $ds=0$  

		 \pv 50mm 
		 {\defi Geodesics} 
	 
		The curves that make the \og distance \fg \, between two events in space-time extreme are called geodesics. They verify the differential equations :
		$$\ \ \ \displaystyle{d^2x^k\over d\lambda^2}\ +\ \Gamma_{ij}^k{dx^i\over d\lambda}{dx^j\over d\lambda}\, =\,0\ \ \ k\in\{1,2,3,4\} \ \ \ $$
		with the connection coefficients (Christoffel symbols of second kind)
		$$\ \ \ \displaystyle \Gamma_{jk}^i=\Gamma_{kj}^i={1\over 2}g^{ip}(\,{\partial g_{pk} \over  \partial x^j} +\,{\partial g_{pj} \over  \partial x^k}-\,{\partial g_{kj} \over  \partial x^p})$$ 

		With the function :$\ \ \mathcal L \,(..x^p(\lambda)..,\,..\dot x^q(\lambda)..)\,=\,\displaystyle \sqrt {\epsilon\,g_{ij}\,{ dx^i\over d\lambda}\,{dx^j\over d\lambda} }\,=\,{ds\over d\lambda}\ \  $
		\pv 4mm
		 where $\ \ \epsilon=1$ for timelike curves ($ds^2>0$) and $\epsilon=-1$ for spacelike curves ( $ds^2<0$) 
		
		(for a signature $\ -\ -\ -\ +$) . 


		The curves $\{x^k(\lambda)\}_{k\in\{1,2,3,4\} }$ that make the integral $\displaystyle \int_a^b ds\,=\,\int_{\lambda_a}^{\lambda_b} \mathcal L \,(\lambda)\,d\lambda\,=\,s(b)-s(a)$ 
				
				extreme are the geodesics. \\ 

		To find first integrals of geodesic equations, we can look for solutions \pv 2mm
		to Euler-Lagrange's equations :
		$\ \ \ \displaystyle {d\over d\lambda}\,{\partial\mathcal L \over \partial\dot x^k} -{\partial\mathcal L\over \partial x^k}\,=\,0 \ \ \ k\in\{1,2,3,4\}$  
		 
			\pv 2mm
		
		\hypertarget{ancre08}{{\bidon III.3  Application }} 
		 
		 
		  For Schwarzschild and Kerr metrics the functions $g_{ij}$ are independent of $t$ and $\varphi$  :
			
		$$ \displaystyle\frac{\partial\mathcal L}{\partial t}=0  \ph 5mm  and  \ph 5mm \displaystyle\frac{\partial\mathcal L}{\partial\varphi}=0 $$
		
		$$\Longrightarrow \ph 5mm \displaystyle\frac{\partial\mathcal L}{\partial\dot t}= constante1 \ph 5mm  and  \ph 5mm \displaystyle\frac{\partial\mathcal L}{\partial\dot\varphi}=  constante2 $$ 
		\pv 3mm
		By imposing $\displaystyle\theta =\frac{\pi}{2}$ the combination of the two previous relationships gives the equations $S1,S2,K1,K2$.
		 

		\hypertarget{ancre09}{{\bidon III.4  Runge-Kutta method of order 4 } }
		 		 

		  Method for numerically solving the differential equation :
        $~~~\displaystyle\frac{d^2r}{d\lambda^2}(r)= f(r)$ 
				
				with initial values $\ph 3mm \lambda=\lambda_0~~~,~~~r=r_0~~ $ and $~~ \displaystyle\frac{dr}{d\lambda}(r_0)$.  

		We calculate the values $r_n$ , $y'_n=\displaystyle\frac{dr}{d\lambda}(r_n)$ and $\varphi_n$ starting from $(r_0\ ,\displaystyle\frac{dr}{d\lambda}(r_0))\ ,\ \varphi_0)$
		with a $h$ step for the variable $\lambda$.	
			\pv -5mm
		$$k_1 = f(r_n) \ \ \ , \ \ \  k_2 = f (r_n + \frac{h}{2}\,y'_n) \ \ \ ,\ \ \ k_3 = f(r_n + \frac{h}{2}\,y'_n+\frac{h^2}{4}\,k_1)\ \ \ ,\ \ \ k_4 = f(r_n + h\,y'_n+\frac{h^2}{2}\,k_2)$$
        $$r_{n+1} = r_n +h\,y'_n+\frac{h^2}{6}\,(k_1+k_2+k_3)$$
        $$y'_{n+1}=y'_n+\frac{h}{6}\,(k_1+2\,k_2+2\,k_3+k_4)$$
		$$ \varphi_{n+1}=\varphi_n+d\varphi ~~~ see ~~~ S2~~~ or ~~~ K2$$ \pv -2mm
		see \href{./documents/Rungekutta.pdf}{\textcolor{bleu}{here} }
		
	\pv 50mm	
		
		\hypertarget{ancre10}{{\bidon III.5  Physical speed}} \pv 8mm

\begin{wrapfigure}[9]{r}{0.55\textwidth}   % [9] indique le nombre de lignes pour la figure
 \vspace{-2.7cm}
  \centering
    \includegraphics[width=0.40\textwidth]{../Images/position_vitesse_physique.jpg}
  \end{wrapfigure}

The physical speed of a moving body passing at the point $r,\theta,\varphi$ of the space-time is the one which would be measured in an orthonormal r\'ef\'erential
attached to an observer (not to be confused with the "distant observer") located at the point  $r,\theta,\varphi$.

With  $\theta=\displaystyle\frac{\pi}{2} $ we note :

-\ \ $r$ and $\varphi$ the polar coordinates of a point $P$ of the trajectory (initial values $r_0$ et $\varphi_0$).

-\ \ $\phi$  the polar angle of the physical velocity vector (initial value $\phi_0$).

-\ \ $V$ the physical speed module (\fcolorbox{black}{green}{$v_0=V(r_0)\ph 5mm \phi_0=\phi(r_0)$}).

-\ \  $V_r\,=V\,cos(\phi)$ et $V_\varphi\,=\,V\,\,sin(\phi)$ the algebraic values of the physical radial velocity and the physical tangential velocity 
\pv 5mm
		
				\hypertarget{ancre11}{{\bidon III.6  Physical speed in Schwarzschild metric}} \pv 3mm
The Scharwzschild tensor (internal and external) can be written as :
 $$ds^2= g_{t,t}(r)\,dt^2\,+\,g_{r,r}(r)\,dr^2\,+ \,g_{\theta,\theta}(r)\,d{\theta}^2\,+\, g_{\varphi,\varphi}(r)\,d{\varphi}^2 $$
in a system of coordinates $\{t,r,\theta,\varphi\}$.

\begin{wrapfigure}{l}{0.40\textwidth}
 \vspace{-30pt}
  \begin{center}
    \includegraphics[width=0.3\textwidth]{../Images/vitesse_locale.jpg}
  \end{center}
 % \vspace{-50pt}
  %\caption{A gull}
  %\vspace{-10pt}
\end{wrapfigure}

\ph 10mm At point $M$, the observer (zero mass) (geodesic $r, \theta, \varphi$ constants, with a four-speed $\vec {u_0}$), 
determine the physical speed $\vec {V}$ from a particle with a four-speed $\vec u$.	

The generic basis of the tangent vector space in $M$ shall be noted :
 $\{\vec {\partial_t}, \vec {\partial_r}, \vec {\partial_\theta}, \vec {\partial_\varphi}  \}$ 
and an orthonormal basis $\{\vec {e^t},\vec {e^r}, \vec {e^\theta}, \vec {e^\varphi} \}$, so the speed $\vec {V}$ is written as    :

 $$\vec {V}\,=\,V^r \vec {\partial_r} \,+\,V^\theta \vec {\partial_\theta}\,+\,V^\varphi \vec {\partial_\varphi}$$
 $$=\,V_r \vec {e^r}\,+\,V_\theta \vec {e^\theta} ,+\,V_\varphi \vec {e^\varphi} $$
With the obvious orthonormal basis : $\{\vec {e^t}=\frac{1}{\sqrt{  g_{t,t} }}\,\vec {\partial_t},\,\vec {e^r}=
\frac{1}{\sqrt{  g_{r,r} }}\,\vec {\partial_r},\,\vec {e^\theta}=
\frac{1}{\sqrt{  g_{\theta,\theta} }}\,\vec {\partial_\theta},\,\vec {e^\varphi}=\frac{1}{\sqrt{  g_{\varphi,\varphi} }}\,\vec {\partial_\varphi} \} $
\pv 3mm
so : $\ph 10mm V_r^2\,=\,g_{r,r}{V^r}^2 \ph 10mmV_\theta^2\,=\,g_{\theta,\theta}{V^\theta}^2 \ph10mm V_\varphi^2\,=\,g_{\varphi,\varphi}{V^\varphi}^2$
\pv 3mm
The 4-speed of the local observer $\vec{u_0}=(\displaystyle\frac{dt}{d\tau_{loc}},\frac{dr}{d\tau_{loc}},\frac{d\theta}{d\tau_{loc}},\frac{d\varphi}{d\tau_{loc}})
=(\frac{c}{\sqrt{  g_{t,t} }},0,0,0)  $ and  the 4-speed of the particle $\vec{u}=(\displaystyle\frac{dt}{d\lambda},\frac{dr}{d\lambda},
\frac{d\theta}{d\lambda},\frac{d\varphi}{d\lambda})$ (parameters $\lambda$ for photons and $\tau$ for the proper time of other particles).

With hypothesis $ \vec u\,=\,\Gamma(\vec {u_0}\,+\,\vec {V})$ and the scalar product $\vec {u_0} . \vec {V} = 0$ 
 the square of the physical speed is equal to : $V_r^2\,+\,V_\theta^2\,+\,V_\varphi^2$ .

Calculation for $\Gamma$ with $\vec u.\vec {u_0}=\Gamma(\vec {u_0}.\vec {u_0}+\vec {u_0}.\vec {V})$ so : 

$$g_{\alpha,\beta}u^\alpha u_0^\beta= g_{t,t}\displaystyle\frac{c}{\sqrt{  g_{t,t} }}\frac{dt}{d\lambda}   = \Gamma g_{\alpha,\beta}u_0^\alpha u_0^\beta=\Gamma c^2
\ph 10mm\Longrightarrow \ph 10mm \Gamma= \frac{\sqrt{  g_{t,t} }}{c}\frac{dt}{d\lambda}        $$.

On the other hand : 
 $$\ph -5mm \Gamma V^r=\displaystyle \frac{dr}{d\lambda}\ \ et\ \  V_r^2=\frac{  g_{r,r}  }{\Gamma^2}({\frac{dr}{d\lambda}})^2 
\ph 10mm \Gamma V^\theta=\displaystyle \frac{d\theta}{d\lambda}\ \ et\ \ V_\theta^2=\frac{  g_{\theta,\theta}   }{\Gamma^2}({\frac{d\theta}{d\lambda}})^2
 \ph 10mm \Gamma V^\varphi=\displaystyle \frac{d\varphi}{d\lambda}\ \ et\ \ V_\varphi^2=\frac{  g_{\varphi,\varphi}   }{\Gamma^2}({\frac{d\varphi}{d\lambda}})^2$$

So, expression for the \fcolorbox{black}{green}{physical speed squared} :

$$\displaystyle\frac{c^2}{ \mid g_{t,t}\mid   } \left[   g_{r,r}   (\frac{dr}{d\lambda})^2 \,+\,   g_{\theta,\theta}   (\frac{d\theta}{d\lambda})^2\,+
\,   g_{\varphi,\varphi}   (\frac{d\varphi}{d\lambda})^2 \right] (\frac{d\lambda}{dt})^2 $$

\centerline{ = \fcolorbox{black}{green}{$\,\displaystyle \frac{c^2}{  \mid g_{t,t} \mid  } \left[   g_{r,r}   (\frac{dr}{dt})^2 \,+\,   g_{\theta,\theta}   (\frac{d\theta}{dt})^2\,+\,   g_{\varphi,\varphi}   (\frac{d\varphi}{dt})^2 \right]     $}  }
\pv 3mm
This method is the formatting of handwritten notes from {\bidon Eric Gougoulhon}   (presented a conference in Montpellier in 2017).
\pv 5mm
{\bidon A - External Scharwzschild metric - particular case $\theta=\displaystyle\frac{\pi}{2} $} 
        \pv 3mm
				Metric outside a centro-symmetrical star, of mass $M$, without rotation.
\pv 2mm
				The coefficients of the metric are : 
				$$\displaystyle g_{t,t}= -c^2(1-\frac{r_s}{r}) \ph 10mm g_{r,r}= \frac{1}{(1-\displaystyle\frac{r_s}{r})} \ph 10mm
				g_{\theta,\theta}=0 \ph 10mm g_{\varphi,\varphi}=r^2 $$
								\pv 3mm
{\titre A.1  Photons} 

First integrals : \ph 30mm $ \displaystyle{dt\over d\lambda}= \displaystyle{E\over 1-\displaystyle{r_s\over r}}\ph 20mm {d\varphi\over d\lambda} = {c\,L\over r^2}$     
$${d\varphi\over dt}\ =\ {1\over E}\ (1-{r_s\over r})\ {c\,L\over r^2} \ph 20mm
({dr \over dt})^2 \ =\ {c^2\over E^2}\ (1-{r_s\over r})^2 \left[E^2-\displaystyle (1-{r_s\over r}){L^2\over r^2}\right] $$ 
Calculation for the  \fcolorbox{black}{green}{physical speed squared} :
$$ \frac{1}{(1-\displaystyle\frac{r_s}{r})^2}\, \left[ ({dr \over dt})^2 \ + \ ({1-\frac{r_s}{r}})\,r^2\,({d\varphi\over dt})^2 \right]\ =\ \fcolorbox{black}{green}{$c^2$} $$
\pv 3mm
\hypertarget{A1photon}{\fcolorbox{black}{white}{\defi Initial values  III.6-A.1 }}\pv -3mm
  $$V_r(r_0)=c\,cos(\phi_0) \ph 20mm  V_\varphi(r_0)=c\,sin(\phi_0)$$
  so  \pv -8mm
  $$E=1    \ph 20mm L=\displaystyle\frac{r_0\,sin(\phi_0)}{\sqrt{1-\displaystyle\frac{r_s}{r_0}}} $$

{\titre A.2  Other particles}

First integrals : \ph 30mm $\displaystyle{dt\over d\tau}=\displaystyle{E\over 1-\displaystyle{r_s\over r}}\ph 20mm {d\varphi\over d\tau} = {c\,L\over r^2}$ 
$${d\varphi\over dt}\ =\ {1\over E}\ (1-{r_s\over r})\ {c\,L\over r^2} \ph 20mm
({dr \over dt})^2 \ =\ {c^2\over E^2}\ (1-{r_s\over r})^2 \left[E^2-\displaystyle (1-{r_s\over r})(1+{L^2\over r^2}) \right]       $$
Calculation for the \fcolorbox{black}{green}{physical speed squared} :
$$ \frac{1}{(1-\displaystyle\frac{r_s}{r})^2}\, \left[ ({dr \over dt})^2 \ + \ ({1-\frac{r_s}{r}})\,r^2\,({d\varphi\over dt})^2 \right]
\ =\ \fcolorbox{black}{green}{$\displaystyle\frac{c^2}{E^2}\,\left[E^2-(1-\displaystyle\frac{r_s}{r})\right]$} $$
\pv 5mm
\hypertarget{A2massif}{\fcolorbox{black}{white}{\defi Initial values  III.6-A.2}}
 $$V_r(r_0)=v_0\,cos(\phi_0) \ph 20mm  V_\varphi(r_0)=v_0\,sin(\phi_0)$$
 so
   $$E=\displaystyle\frac{\sqrt{1-\displaystyle\frac{r_s}{r_0}}}{\sqrt{1-\displaystyle\frac{v_0^2}{c^2}}} \ph 10mm L=\frac{v_0}{c}\frac{ r_0 E \sin(\phi_0)}{ \sqrt{1-\displaystyle\frac{r_s}{r_0}}} $$

{\bidon B - Internal Scharwzschild metric - particular case $\theta=\displaystyle\frac{\pi}{2} $} 
\pv 3mm
Metric inside a star of radius $R$, mass $M$, constant density , without rotation.
        \pv 2mm
				The coefficients of the metric are : 
				$$\displaystyle g_{t,t}=- c^2 \left [\frac{3}{2}\ \sqrt{ 1-\frac{r_s}{ R}}\ -\ \frac{1}{2}\ \sqrt{1-\frac{r^2 r_s}{R^3}}\ \right ]^2
				\ph 10mm g_{r,r}=\displaystyle\frac{1}{1-\displaystyle\frac{r^2 r_s}{R^3} }\ph 10mm
				g_{\theta,\theta}=0 \ph 10mm g_{\varphi,\varphi}=r^2 $$
								\pv 3mm
	 With \ph 10mm $\alpha (r)= 1-\displaystyle\frac{r^2 r_s}{R^3}$ \ph 20mm $\beta (r)=\frac{3}{2}\ \sqrt{\displaystyle 1-\frac{r_s}{ R}}\ -
\ \frac{1}{2}\ \sqrt{\displaystyle 1-\frac{r^2 r_s}{R^3}}$ 
\pv 5mm
{\titre B.1  Photons} 

First integrals : \ph 30mm $ \displaystyle \beta(r)^2 {dt\over d\lambda}=E \ph 20mm {d\varphi\over d\lambda} = \frac{c L}{r^2}$
$${d\varphi\over dt}\ =\ {\beta(r)^2\over E}\,{c\,L\over r^2} \ph 20mm ({dr \over dt})^2 \ 
=\ {c^2\over E^2}\,\alpha(r)\,\beta(r)^4 \left[\frac{E^2}{\beta(r)^2}\,-\,\frac{L^2}{r^2}\right] $$ 
Calculation for the \fcolorbox{black}{green}{physical speed squared} :
$$ \frac{1}{\beta(r)^2 \alpha(r)}\, \left[\displaystyle ({dr \over dt})^2 \ + \ \alpha(r) r^2\,({d\varphi\over dt})^2 \right]\ =\ \fcolorbox{black}{green}{$c^2$}  $$
\pv 3mm
\hypertarget{B1photon}{\fcolorbox{black}{white}{\defi Initial values III.6-B.1}}
  $$V_r(r_0)=c\,cos(\phi_0) \ph 20mm  V_\varphi(r_0)=c\,sin(\phi_0)$$
  so
  $$E=1    \ph 20mm L=\displaystyle\frac{r_0\,sin(\phi_0)}{\beta(r_0)} $$
	\pv 50mm
{\titre B.2  Other particles}

First integrals : \ph 30mm $ \displaystyle \beta(r)^2 {dt\over d\tau}=E \ph 20mm {d\varphi\over d\tau} = \frac{c L}{r^2}$
$${d\varphi\over dt}\ =\ {\beta(r)^2\over E}\,{c\,L\over r^2} \ph 20mm ({dr \over dt})^2 \ 
=\ {c^2\over E^2}\,\alpha(r)\,\beta(r)^4 \left[\frac{E^2}{\beta(r)^2}\,-\,\frac{L^2}{r^2}\,-1\right] $$ 
Calculation for the \fcolorbox{black}{green}{physical speed squared} :
$$ \frac{1}{\beta(r)^2 \alpha(r)}\, \left[\displaystyle ({dr \over dt})^2 \ + \ \alpha(r) r^2\,({d\varphi\over dt})^2 \right]\ 
=\ \fcolorbox{black}{green}{$\displaystyle\frac{c^2}{E^2}\,\left[E^2-\beta(r)^2\right]$}$$
\pv 3mm
\hypertarget{B2massif}{\fcolorbox{black}{white}{\defi Initial values III.6-B.2}}
  $$V_r(r_0)=v_0\,cos(\phi_0) \ph 20mm  V_\varphi(r_0)=v_0\,sin(\phi_0)$$
  d'o\`u
     $$E=\displaystyle\frac{\beta(r_0)}{\sqrt{1-\displaystyle\frac{v_0^2}{c^2}}} 
		\ph 20mm L=\frac{v_0}{c}\frac{r_0 \sin{\phi_0}}{\sqrt{1-\displaystyle\frac{v_0^2}{c^2}}}  $$


		\hypertarget{ancre12}{{\bidon III.7  Physical speed in Kerr metric }} \pv 3mm
		The Kerr tensor can be written as :
 $$ds^2= g_{t,t}(r)\,dt^2\,+\,g_{t,\varphi}(r)\,dt\,d\varphi\,+g_{\varphi,t}(r)\,d\varphi\,dt\,+\,g_{r,r}(r)\,dr^2\,+ \,g_{\theta,\theta}(r)\,d{\theta}^2\,+\, g_{\varphi,\varphi}(r)\,d{\varphi}^2 $$
in a system of coordinates $\{t,r,\theta,\varphi\}$.

\begin{wrapfigure}{l}{0.40\textwidth}
 \vspace{-30pt}
  \begin{center}
    \includegraphics[width=0.3\textwidth]{../Images/vitesse_locale.jpg}
  \end{center}
 % \vspace{-50pt}
  %\caption{A gull}
  %\vspace{-10pt}
\end{wrapfigure}

\ph 10mm At point $M$, the observer (zero mass) (geodesic $r, \theta, \varphi$ constants,  with a four-speed $\vec {u_0}$), 
determine the physical speed $\vec {V}$ from a particle  with a four-speed $\vec u$.	

The generic basis of the tangent vector space in $M$ shall be noted :
 $\{\vec {\partial_t}, \vec {\partial_r}, \vec {\partial_\theta}, \vec {\partial_\varphi}  \}$ 
and an orthonormal basis $\{\vec {e^t},\vec {e^r}, \vec {e^\theta}, \vec {e^\varphi} \}$, so the speed $\vec {V}$ is written as :

 $$\vec {V}\,=\,V^r \vec {\partial_r} \,+\,V^\theta \vec {\partial_\theta}\,+\,V^\varphi \vec {\partial_\varphi}$$
 $$=\,V_r \vec {e^r}\,+\,V_\theta \vec {e^\theta} ,+\,V_\varphi \vec {e^\varphi} $$
 With the orthonormal basis : 
$$\{\ph 5mm \vec {e^t}=\frac{1}{\sqrt{  g_{t,t} }}\,\vec {\partial_t} \ph 10mm \vec {e^r}=
\frac{1}{\sqrt{  g_{r,r} }}\,\vec {\partial_r}\ph 10mm\vec {e^\theta}=
\frac{1}{\sqrt{  g_{\theta,\theta} }}\,\vec {\partial_\theta}\,$$
$$\vec {e^\varphi}=\frac{-g_{t,\varphi}}{g_{t,t}\,\sqrt{  \displaystyle\frac {g_{t,t}\,g_{\varphi,\varphi}\,-\,g_{t,\varphi}\,g_{\varphi,t}}{g_{t,t}} }}\,\vec {\partial_t} 
\,+\,\frac{{1}}{\sqrt{  \displaystyle\frac {g_{t,t}\,g_{\varphi,\varphi}\,-\,g_{t,\varphi}\,g_{\varphi,t}}{g_{t,t}} }}\,\vec {\partial_\varphi} \ph 5mm \} $$

so : $\ph 10mmV_r^2\,=\, g_{r,r}{V^r}^2 \ph 10mm V_\theta^2\,=\,g_{\theta,\theta}{V^\theta}^2 \ph 10mmV_\varphi^2\,=\,\displaystyle\frac {(g_{t,t}\,g_{\varphi,\varphi}\,-\,g_{t,\varphi}\,g_{\varphi,t})}{g_{t,t}} {V^\varphi}^2$
\pv 3mm
The 4-speed of the local observer $\vec{u_0}=(\displaystyle\frac{dt}{d\tau_{loc}},\frac{dr}{d\tau_{loc}},\frac{d\theta}{d\tau_{loc}},\frac{d\varphi}{d\tau_{loc}})
=(\frac{c}{\sqrt{  g_{t,t} }},0,0,0)  $ and  the 4-speed of the particle $\vec{u}=(\displaystyle\frac{dt}{d\lambda},\frac{dr}{d\lambda},
\frac{d\theta}{d\lambda},\frac{d\varphi}{d\lambda})$ ( parameters $\lambda$ for photons and $\tau$ for the proper time of other particles).

With hypothesis $ \vec u\,=\,\Gamma(\vec {u_0}\,+\,\vec {V})$ and the scalar product $\vec {u_0} . \vec {V} = 0$ 
 the square of the physical speed is equal to : $V_r^2\,+\,V_\theta^2\,+\,V_\varphi^2$ .

Calculation for $\Gamma$ with $\vec u.\vec {u_0}=\Gamma(\vec {u_0}.\vec {u_0}+\vec {u_0}.\vec {V})$ so : 
$$ g_{\alpha,\beta}u^\alpha u_0^\beta= g_{t,t}\displaystyle\frac{c}{\sqrt{  g_{t,t} }}\frac{dt}{d\lambda} +g_{t,\varphi}\displaystyle\frac{c}{\sqrt{  g_{t,t} }}\frac{d\varphi}{d\lambda}  = \Gamma g_{\alpha,\beta}u_0^\alpha u_0^\beta=\Gamma c^2 \ph 3mm\Longrightarrow \ph 3mm  \Gamma= \frac{1}{c} [ \sqrt{g_{t,t}}\frac{dt}{d\lambda} +\frac{g_{t,\varphi}}{\sqrt{g_{t,t} }}\frac{d\varphi}{d\lambda}  ]    $$

On the other hand : 
 $$ \Gamma V^r=\displaystyle \frac{dr}{d\lambda}\ \ et\ \  V_r^2=\frac{  g_{r,r}  }{\Gamma^2}({\frac{dr}{d\lambda}})^2 
\ph 20mm \Gamma V^\theta=\displaystyle \frac{d\theta}{d\lambda}\ \ et\ \ V_\theta^2=\frac{  g_{\theta,\theta}   }{\Gamma^2}({\frac{d\theta}{d\lambda}})^2$$
 $$ \Gamma V^\varphi=\displaystyle \frac{d\varphi}{d\lambda}\ \  et \ \ 
 V_\varphi^2=\displaystyle\frac {(g_{t,t}\,g_{\varphi,\varphi}\,-\,g_{t,\varphi}\,g_{\varphi,t})}{\Gamma^2\,g_{t,t}}({\frac{d\varphi}{d\lambda}})^2$$

So, expression for the \fcolorbox{black}{green}{physical speed squared} :

$$\displaystyle\frac{c^2}{(\sqrt{g_{t,t}}\displaystyle\frac{dt}{d\lambda} +\frac{g_{t,\varphi}}{\sqrt{g_{t,t} }}\frac{d\varphi}{d\lambda})^2} \left[   g_{r,r}   (\frac{dr}{d\lambda})^2 \,+\,   g_{\theta,\theta}   (\frac{d\theta}{d\lambda})^2\,+\,\displaystyle\frac {(g_{t,t}\,g_{\varphi,\varphi}\,-\,g_{t,\varphi}\,g_{\varphi,t})}{g_{t,t}} (\frac{d\varphi}{d\lambda})^2 \right]  $$

\centerline{ = \fcolorbox{black}{green}{$\,\displaystyle \frac{c^2}{(\sqrt{g_{t,t}}+\displaystyle\frac{g_{t,\varphi}}{\sqrt{g_{t,t} }}\frac{d\varphi}{dt})^2}   \left[   g_{r,r}   (\frac{dr}{dt})^2 \,+\,           g_{\theta,\theta}(\frac{d\theta}{dt})^2\,+\,\displaystyle\frac {(g_{t,t}\, g_{\varphi,\varphi}\,-\, g_{t,\varphi}\, g_{\varphi,t})}{g_{t,t}}   (\frac{d\varphi}{dt})^2 \right]     $}  }
\pv 3mm
This is an application of the written method in handwritten notes from {\bidon Eric Gougoulhon} (presented a conference in Montpellier in 2017).
\pv 5mm
{\bidon A - Kerr metric} 
        \pv 3mm
				Metric outside a centro-symmetric \underline{black hole}, of mass $M$, with constant rotation.
\pv 2mm
				The coefficients of the metric are : 
				$$ \displaystyle g_{t,t}= -c^2(1-\frac{r_s\,r}{\rho^2})\ph 10mm g_{t,\varphi}=g_{\varphi,t}=-\frac{c\, r_s\, a\, r}{\rho^2}\sin^2(\theta)$$
			\pv -3mm	$$ g_{r,r}= \frac{\rho^2}{\Delta}   \ph 6mm g_{\theta,\theta}=\rho^2  \ph 6mm g_{\varphi,\varphi}=\left[r^2+a^2+\displaystyle\frac{r_s\, a^2\, r}{\rho^2}\sin^2(\theta)\right]\sin^2(\theta) $$
			\pv 2mm	$$\rho^2=r^2+a^2\cos^2\theta\ph 10mm\Delta=r^2-r_s\,r+a^2$$
								\pv 3mm
{\titre A.1  Photons - Particular case $\theta=\frac{\Pi}{2}$} 
\pv 3mm
First integrals :
$$\displaystyle{dt\over d\lambda}={1\over\Delta}\,\left[\,(r^2+a^2+{r_s\over r}\,a^2)\,E-{r_s\,a\over r}\,L\,\right]
\ph 20mm \displaystyle{d\varphi\over d\lambda} = {c\over\Delta}\,\left[\,{r_s\,a\over r}\,E+(1-{r_s\over r})\,L\,\right]$$

 $$\displaystyle {d\varphi\over dt}\ =\ \displaystyle\frac{c\,\left[\,\displaystyle{r_s\,a\over r}\,E+(1-\displaystyle{r_s\over r})\,L\,\right]}{\left[\,(r^2+a^2+\displaystyle{r_s\over r}\,a^2)\,E-\displaystyle{r_s\,a\over r}\,L\,\right]}$$
\pv 5mm
$$({dr \over dt})^2 \ =\ c^2\ \frac{\left[E^2+\displaystyle\frac{(a^2\,E^2-L^2)}{r^2}+r_s\,\displaystyle\frac{(L-a\,E)^2}{r^3}\right](a^2+r^2-r\,r_s)^2}{\left[(r^2+a^2+\displaystyle\frac{r_s}{r}\,a^2)\,E-\frac{r_s\,a}{r}\,L\right]^2  }$$ 

Calculation for the \fcolorbox{black}{green}{physical speed squared} :
$$ \frac{(1-\displaystyle\frac{r_s}{r})}{\left[ (1-\displaystyle\frac{r_s}{r}) + \displaystyle\frac{r_s\,a}{c\,r}\,({d\varphi\over dt}) \right]^2} \left[ \frac{r^2}{\Delta}\,({dr \over dt})^2 \ + \ \displaystyle\frac{\Delta}{(1-\displaystyle\frac{r_s}{r})}\,({d\varphi\over dt})^2 \right]\ =\ \fcolorbox{black}{green}{$c^2$} $$

\hypertarget{Kerrphoton}{\fcolorbox{black}{white}{\defi Initial values III.7-A.1}}\pv 3mm
  $$V_r(r_0)=c\,cos(\phi_0) \ph 20mm  V_\varphi(r_0)=c\,sin(\phi_0)$$
  d'o\`u
  $$E=1   \ph 20mm L=\frac{1}{(r_0-r_s)}\left[r_0\,sin(\phi_0)\, \sqrt{\Delta(r_0)}\,-\,a\,r_s \right]   $$
\pv 5mm
{\titre A.2  Other particles - Particular case $\theta=\frac{\Pi}{2}$} 
\pv 3mm
First integrals :
$$\displaystyle{dt\over d\tau}={1\over\Delta}\,\left[\,(r^2+a^2+{r_s\over r}\,a^2)\,E-{r_s\,a\over r}\,L\,\right]
\ph 20mm \displaystyle{d\varphi\over d\tau} = {c\over\Delta}\,\left[\,{r_s\,a\over r}\,E+(1-{r_s\over r})\,L\,\right]$$

 $$\displaystyle {d\varphi\over dt}\ =\ \displaystyle\frac{c\,\left[\,\displaystyle{r_s\,a\over r}\,E+(1-\displaystyle{r_s\over r})\,L\,\right]}{\left[\,(r^2+a^2+\displaystyle{r_s\over r}\,a^2)\,E-\displaystyle{r_s\,a\over r}\,L\,\right]}$$
\pv 5mm
$$({dr \over dt})^2 \ =\ c^2\ \frac{\left[E^2-1+\displaystyle\frac{r_s}{r}+\displaystyle\frac{(a^2(E^2-1)-L^2)}{r^2}+r_s\,\displaystyle\frac{(L-a\,E)^2}{r^3}\right](a^2+r^2-r\,r_s)^2}{\left[(r^2+a^2+\displaystyle\frac{r_s}{r}\,a^2)\,E-\frac{r_s\,a}{r}\,L\right]^2  }$$ 

Calculation for the \fcolorbox{black}{green}{physical speed squared} :
$$ \frac{(1-\displaystyle\frac{r_s}{r})}{\left[ (1-\displaystyle\frac{r_s}{r}) + \displaystyle\frac{r_s\,a}{c\,r}\,({d\varphi\over dt}) \right]^2} \left[ \frac{r^2}{\Delta}\,({dr \over dt})^2 \ + \ \displaystyle\frac{\Delta}{(1-\displaystyle\frac{r_s}{r})}\,({d\varphi\over dt})^2 \right]\ 
\ =\ \fcolorbox{black}{green}{$\displaystyle\frac{c^2}{E^2}\,\left[E^2-(1-\displaystyle\frac{r_s}{r})\right]$} $$

\hypertarget{Kerrmassif}{\fcolorbox{black}{white}{\defi Initial values III.7-A.2}}
  $$V_r(r_0)=v_0\,cos(\phi_0) \ph 20mm  V_\varphi(r_0)=v_0\,sin(\phi_0)$$
  so     
$$E=c\,\sqrt{\frac{r_0-r_s}{r_0\,(c^2-v_0^2)}}  \ph 20mm L=\displaystyle\frac{-1}{\sqrt{(c^2-v_0^2)(r_0-r_s)}}\,\left[\frac{a\,c\,r_s}{\sqrt{r_0}}\,-\,v_0\,sin(\phi_0)\,\sqrt{r_0\,\Delta(r_0)} \right]$$

		
		
		
		
		
		
		
		
		
		
		
		
		
		
		
		
		
		
		
		
		
		\end{document}
